\begin{minipage}{0.63\textwidth}
    \parskip=1em
    \section*{セキュリティモジュール:5文字のコード}
    
    \uline{概要}:数字の並び{、}5文字のディスプレイ(上下の矢印を使用して文字を変更できます){、}および\hfill\\{[OK]}ボタンがある鉄板。
\end{minipage}%
\hfill%
\begin{minipage}{0.33\textwidth}
    \includegraphics[width=\textwidth]{images/9.png}
    \vspace*{\fill}
\end{minipage}

\uline{解除方法}:正しい5文字のコードを入力し{、} {[OK]}を押します。\\
左から順番に数字を足していきます。偶数の桁になったら{、}足すのをやめます。(ただし{、}偶数の桁も足します。)下の表で結果を見つけてください。\\
残りの桁について{、}この過程を繰り返します。

例:\\
\hspace*{3em}1112の合計は5になり、文字Aに対応します。\\
\hspace*{3em}1112330は5と6になり、文字AとBに対応します。


\begin{center}
\large
\def\arraystretch{1.5}
\begin{tabular}{|l|l|l}
    \hline
    A - \ 5 \hspace*{1em}  & J - \ 17\hspace*{1em}  & \multicolumn{1}{l|}{S - \ 2\hspace*{1em} }   \\ \hline
    B - \ 6 \hspace*{1em}  & K - \ 21\hspace*{1em}  & \multicolumn{1}{l|}{T - \ 7\hspace*{1em} }   \\ \hline
    C - \ 27\hspace*{1em}  & L - \ 8 \hspace*{1em}  & \multicolumn{1}{l|}{U - \ 25\hspace*{1em} } \\ \hline
    D - \ 12\hspace*{1em}  & M - \ 14\hspace*{1em}  & \multicolumn{1}{l|}{V - \ 15\hspace*{1em} } \\ \hline
    E - \ 0 \hspace*{1em}  & N - \ 10\hspace*{1em}  & \multicolumn{1}{l|}{W - \ 16\hspace*{1em} } \\ \hline
    F - \ 11\hspace*{1em}  & O - \ 3 \hspace*{1em}  & \multicolumn{1}{l|}{X - \ 19\hspace*{1em} } \\ \hline
    G - \ 26\hspace*{1em}  & P - \ 22\hspace*{1em}  & \multicolumn{1}{l|}{Y - \ 20\hspace*{1em} } \\ \hline
    H - \ 13\hspace*{1em}  & Q - \ 18\hspace*{1em}  & \multicolumn{1}{l|}{Z - \ 24\hspace*{1em} } \\ \hline
    I - \ 4 \hspace*{1em}  & R - \ 9 \hspace*{1em}  &                             \\\cline{1-2}
\end{tabular}
\end{center}


「まだ生きている爆発物処理班」の為のヒント:
\begin{itemize}
    \item[$\bullet$] 数列にはちょうど5つの偶数があります。
    \item[$\bullet$] 「ゼロ」も偶数!
\end{itemize}
