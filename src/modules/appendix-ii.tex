\section*{付録II:起爆剤の種類}

TiNT博士は比較的小さな起爆剤を爆弾内に配置します。その起爆剤が主火薬を爆発させます。最も一般的な起爆剤は次のとおりです。




\bgroup
\def\arraystretch{1.3}
\begin{tabular}{|>{\centering}p{0.3\textwidth}|p{0.6\textwidth}|}
\hline
種類  & \hspace*{\fill}特性\hspace*{\fill} \\ \hline
{\Large C-4} & \begin{tabular}{@{}l@{}}主成分:RDX\\ 化学式:\ce{C3H6N6O6}\\ 化合物分類:脂肪族\\ 相対有効係数:1.6*\\ 爆発速度:8750 m/s\end{tabular} \\ \hline
{\Large Semtex} & \begin{tabular}{@{}l@{}}主成分:PETN\\ 化学式:\ce{C5H8N4O12}\\ 化合物分類:脂肪族\\ 相対有効係数:1.66*\\ 爆発速度:8400 m/s\end{tabular} \\ \hline
{\Large ダイナマイト} & \begin{tabular}{@{}l@{}}主成分:ニトログリセリン\\ 化学式:\ce{C3H5N3O6}\\ 化合物分類:脂肪族\\ 相対有効係数:1.5*\\ 爆発速度:7700 m/s\end{tabular}  \\ \hline
{\Large TNT} & \begin{tabular}{@{}l@{}}主成分:トリニトロトルエン\\ 化学式:\ce{C7H5N3O6}\\ 化合物分類:芳香\\ 相対有効係数:1.0*\\ 爆発速度:6900 m/s\end{tabular} \\ \hline
{\Large 即席爆発物}  & \begin{tabular}{@{}l@{}}主成分:TATP\\ 化学式:\ce{C9H18O6}\\ 化合物分類:脂肪族\\ 相対有効係数:0.83*\\ 爆発速度:5300 m/s\end{tabular}  \\ \hline
\end{tabular}
\egroup

* TNT 1 kgに対して
